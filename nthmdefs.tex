\usepackage[amsmath,thmmarks,framed]{ntheorem}
% get it from : http://user.informatik.uni-goettingen.de/~may/Ntheorem/

\usepackage{color}
\usepackage{framed}

%\usepackage[pdftex]{graphicx}
\usepackage{graphicx}

\newcommand{\Zs}{\mathbb{Z}}
\newcommand{\Ns}{\mathbb{N}}
\newcommand{\Rs}{\mathbb{R}}


\newcommand{\tq}[1][]{\text{ tel#1 que }}

\newcommand{\HRule}{\rule[0mm]{0.4\linewidth}{0.5pt}}

\makeatletter
\newtheoremstyle{nonumberplainperso}
  {\item[\hskip\labelsep \theorem@headerfont  \textcolor{\currentcolor}{\underline{ \textcolor{black}{##1\ }}} ]}%
  {\item[\hskip\labelsep \theorem@headerfont \textcolor{\currentcolor}{\underline{ \textcolor{black}{ ##1\  : ##3}}}]}
\newtheoremstyle{nonumberplainperso2}
  {\item[\hskip\labelsep \theorem@headerfont  \textcolor{\currentcolor}{\underline{ \textcolor{black}{##1\ :}}} ]}%
  {\item[\hskip\labelsep \theorem@headerfont \textcolor{\currentcolor}{\underline{ \textcolor{black}{ ##1\  ##3 :}}}]}
\newtheoremstyle{nonumberbreakperso}
  {\item[\rlap{\vbox{\hbox{\hskip\labelsep \theorem@headerfont
           \textcolor{\currentcolor}{\underline{ \textcolor{black}{##1 }}}\theorem@separator}\hbox{\strut}}}]}%
  {\item[\rlap{\vbox{\hbox{\hskip\labelsep \theorem@headerfont
           \textcolor{\currentcolor}{\underline{ \textcolor{black}{##1\ : ##3 }}} \theorem@separator}\hbox{\strut}}}]}
\newtheoremstyle{nonumberbreakperso2}
  {\item[\rlap{\vbox{\hbox{\hskip\labelsep \theorem@headerfont
           \textcolor{\currentcolor}{\underline{ \textcolor{black}{##1 }}}\theorem@separator}\hbox{\strut}}}]}%
  {\item[\rlap{\vbox{\hbox{\hskip\labelsep \theorem@headerfont
           \textcolor{\currentcolor}{\underline{ \textcolor{black}{##1\  ##3 }}} \theorem@separator}\hbox{\strut}}}]}
\makeatother

\definecolor{sepia}{RGB}{70,100,70}
\newcommand{\defcolor}{green}
\newcommand{\thmcolor}{red}
\newcommand{\rmcolor}{blue}
\newcommand{\demcolor}{yellow}
\newcommand{\lemcolor}{magenta}
\newcommand{\propcolor}{cyan}
\newcommand{\quescolor}{violet}
\newcommand{\currentcolor}{\defcolor}

\newcommand{\renewframeperso}[4]{
\renewcommand{\currentcolor}{#1}
\renewcommand{\HRule}{\rule[0mm]{#4\linewidth}{0.5pt}}
\renewcommand*\FrameCommand{{\color{white}\vrule width #2}{\color{#1}\vrule width 1pt  \HRule \hspace{#3}\hspace{-#4\linewidth}  }}
}

\theoremstyle{nonumberbreakperso}
\theoremheaderfont{\normalfont}
\theorembodyfont{\upshape}
\theoremseparator{}
\theoremprework{\renewframeperso{\defcolor}{0pt}{5pt}{0.4}}
\newframedtheorem{definition}{Définition}


\theoremstyle{nonumberbreakperso2}
\theoremheaderfont{\normalfont}
\theorembodyfont{\upshape}
\theoremseparator{}
\theoremprework{\renewframeperso{\thmcolor}{0pt}{5pt}{0.4}}
\newframedtheorem{thm}{Théorème}

\theoremstyle{nonumberplainperso}
\theorembodyfont{\upshape}
\theoremseparator{}
\theoremprework{\renewframeperso{\rmcolor}{15pt}{5pt}{0.2}}
\newframedtheorem{rem}{Remarque}
\theoremprework{\renewframeperso{\rmcolor}{15pt}{5pt}{0.2}}
\newframedtheorem{ex}{Exemple}
\theoremprework{\renewframeperso{\rmcolor}{15pt}{5pt}{0.2}}
\newframedtheorem{exs}{Exemples}


\theoremheaderfont{\itshape}\theorembodyfont{\upshape}
\theoremstyle{nonumberbreakperso}
\theoremseparator{}
\theorembodyfont{\color{sepia}}
\theoremprework{\renewframeperso{\demcolor}{20pt}{5pt}{0.4}}
\newframedtheorem{dem}{Dém}

\theorembodyfont{}
\theoremstyle{nonumberbreakperso}
\theoremprework{\renewframeperso{\lemcolor}{10pt}{5pt}{0.4}}
\newframedtheorem{lem}{Lemme}

\theoremprework{\renewframeperso{\lemcolor}{10pt}{5pt}{0.4}}
\newframedtheorem{cor}{Corrolaire}

\theoremstyle{nonumberplainperso2}
\theoremprework{\renewframeperso{\quescolor}{10pt}{5pt}{0.4}}
\newframedtheorem{ques}{Question}


\theoremstyle{nonumberbreakperso}
\theoremindent0cm
\theoremheaderfont{\upshape}

\theoremprework{\renewframeperso{\propcolor}{5pt}{5pt}{0.4}}
\newframedtheorem{prop}{Proposition}

\theoremprework{\renewframeperso{\propcolor}{5pt}{5pt}{0.4}}
\newframedtheorem{propriete}{Propriété}

