\documentclass[a4paper,10pt]{article}
\usepackage[utf8x]{inputenc}
\usepackage{amsmath}
\usepackage{amssymb}
%opening
\newcommand{\F}{\mathcal{F}}

\begin{document}


\subsection{Formule de Sahlquist}
On définit une construction syntaxique 
$S ::= X \rightarrow POS$
où POS est une formule positive et $\chi ::= T | \perp | \Box_{i_1} \cdots \Box_{i_n}p | \chi \wedge \chi | \Diamond_{j_i} \chi$
La formule générée par S
\begin{thm}{} Soit $\Phi$ une formule de Sahlquist. Il existe une formule du 1er ordre $\alpha$ qui correspond à $\Phi$ sur les cadres.\\
Pour tout cadre $\F$, pour tout monde $m$,
$\F, m \models \Phi$ ssi $\F[x \leftarrow m] \models \alpha(x)$
\end{thm}
\begin{exemple}{1}
 \F, m \models \Phi ssi \forall V, \F, V, m \models \Phi
ssi \forall V, \F, V, w \models p implique \F, V, w \models \Diamond p
ssi \forall V, w \in V(p) implique \F, V, m \models \Diamond p
ssi pour tout V' / V'(p) = {w} implique \F, V, w \models p
ssi \forall V' / V'(p) = \{ w \} implique \F, V, w \models \Diamond p
(car \Diamond p est croissante en p car positive)
(1) ssi \forall V' / V'(p) = w on a \F, V', [x \leftarrow w ] \models \exists y xRy \wedge p(y)
ssi le prédicat p est interprété avec la fonction 
\begin{array}{ll}
 W \longleftarrow \{0, 1\}
 w \mapsto & 1 si v = w
	    & 0 sinon
\end{array}
On a montré que \forall w \in W
\F, w \models p \to \Diamond p ssi F[x \leftarrow w] \models x Rx

On a aussi:
\F \models p \to \Diamond p ssi \forall w, \F[x←w] \models xRx
			    ssi \F \models \forall x, xRx
\end{exemple}

1)Traduire \chi → POS en second ordre de la forme
\forall \vec p, \forall \vec y, [(REL \wedge BOX) → ST_x(POS)
où RES est une conjoniotn de y_i R_k y_j x R_k y_i
   BOX est une conjonction de la forme \forall s, y_i R_{\beta} s → p(s)
2) Retourne la formule \forall \vec y, REL → ST_x(POS) dans laquelle on a remplacé les occurences de p par 
\sigma(p) par \lambda a . y_{i_1} R_{\beta_1} a \vee \cdots \vee y_{i_k} R_{\beta_k} a
où chaque y_{i_j} R_{\beta_j} a provient d'une formule \forall s y_{i_j} R_{\beta_j} s → p(s) qui apparaît dans BOX.

\subsection{Théorème de complétude}

Soit S un ensemble de formules . On écrit \vdash_S \Phi pour dire que \hi est prouvable en utilisant 
\begin{itemize}
 \item les tautologies
 \item l'axiome K
 \item les instances de formules de S
 \item le modus ponens
 \item la nécéssitation
\end{itemize}
Rem : \vdash \Phi; \vdash_{\empty} \Phi
Soit \mathcal{C} une classe de cadres.
\models_{\mathcal{C}} \Phi ssi \forall \F \in \mathcal{C}, \F \models \Phi
\Sigma \models_{\mathcal{C}} \Phi ssi pour tout modèle \mathcal{M}, w basé sur un cadre de \mathcal{C}, on a :
Si \mathcal{M}, w \models \Sigma alors \mathcal{M}, w \models \Phi

\begin{thm}{de complétude de Sahlquist}
Soit S un ensemble de formules de Sahlquist, soit C_s la classe de cadres qui vérifient les propriétés du 1er ordre correspondantes aux formules de S sur les cadres.
Alors  \Sigma \vdash_S \Phi ssi \Sigma \models_{\mathcal{C}} \Phi (complétude forte}
(complétude faible : \vdash_S \Phi ssi \models_S \Phi.)
\end{thm}

Trois techniques pour montrer la décidabilité/compléxité du problème de logique modale.

\subsection{Logique modale S5}
\begin{tabular}{c | c  }
 axiomatique & propriété sur les cadres \\
 \Box p → p & réfléxivité\\
 \Box p → \Box \Box p & transitivité\\
 \lnot\Box p \Box \lnot \Box p & euclidianité\\
 (\Diamond \Box p → \Box p)\\
\end{tabular}
\subparagraph{Application : } Logique epistémique

euclidianité : \forall x, y, z, (x R y \wedge xRz) → y R z

\begin{prop}
 Si R est réfléxive, transitive, alors R euclidienne ssi R symétrique
\end{prop}
\begin{prop}
 \Phi satisfiable dans un modèle où R est une rel d'équivalence 
\Leftrightarrow \Phi satisfiable dans un modèle M = (W, R, V) où R = W \times W
\end{prop}
Dem \vbox{\Leftarrow} : trivial
\vbox{\Rightarrow} Si il existe M, w où M  = (W, R, V) tq
\begin{itemize}
 \item R est une relation d'équivalence
 \item M, w \models \Phi
\end{itemize}












\end{document}
