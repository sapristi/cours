\documentclass[10pt,a4paper]{article}
\usepackage[utf8x]{inputenc}
\usepackage{ucs}
\usepackage{amsmath}
\usepackage{amsfonts}
\usepackage{amssymb}
\usepackage{pgf}
\usepackage{tikz}
\newcommand{\N}{\mathbb{N}}
\newcommand{\Prob}{\mathbb{P}}


\title{MEPS}

\begin{document}
\part{Chaînes de Markox à temps discret}
\section{Définitions et propriétés}
$n \in \N$ : temps (discret).\\
$ \{ X_n \} \in E^{\N} $ : états du système, $E$ l'espace des états, dénombrable.\\
L'état à l'instant $n$ ne dépend que de l'état à l'instant précédent $n-1$.\\

En prenant $\N \times E$ l'espace des états, on peut augmenter la " taille " du présent, jusqu'à inclure tous les états précédents : les chaînes de Markov peuvent en fait modéliser tous les processus stochastiques dont l'état à un instant donné dépend des états passés.\\
 
$\alpha_j = \Prob(X_0 = i_j)$\\


\begin{tikzpicture}[->,>=stealth',shorten >=1pt,auto,node distance=2.8cm,
                    semithick]
                    \node[state] (0)
\end{tikzpicture}




\end{document}

