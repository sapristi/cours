
%\usepackage[pdftex]{graphicx}
\usepackage{graphicx}

\usepackage{color}


\newcommand{\Z}{\mathbb{Z}}
\newcommand{\N}{\mathbb{N}}
\newcommand{\R}{\mathbb{R}}

\newcommand{\HRule}{\rule[2mm]{0.4\linewidth}{0.5pt}}


\makeatletter
\newcommand\parag{%
   \@startsection{paragraph}{4}{0mm}%
      {-\baselineskip}%
      {.5\baselineskip}%
      {\normalfont\normalsize\bfseries}}


\makeatother




\newenvironment{definition}[1]
{	\parag{ \textcolor{green}{	\underline{ \textcolor{black}{Définition : #1}} } } \noindent}
{ \textcolor{green}{\HRule} \bigskip}

\newenvironment{thm}[1]{
	\parag{ 	
		\textcolor{red}{
			\underline{ \textcolor{black}{Théorème : #1}} } }  \noindent} 
			{  \textcolor{red}{\HRule}\bigskip }
			
\newenvironment{rem}[1]{
	\subparagraph{
		\textcolor{blue}{
			\underline{\textcolor{black}{Remarque #1}}}}} {}
			
\newenvironment{dem}[1]{
	\parag{
		\textcolor{yellow}{
			\underline{ \textcolor{black}{Démonstration : #1}} } }} {\textcolor{yellow}{\HRule} \bigskip}	
\newenvironment{lem}[1]{
	\subparagraph{ 	
		\textcolor{magenta}{
			\underline{ \textcolor{black}{Lemme : #1}} } } } {}		
\newenvironment{ex}[1]{
	\subparagraph{ 	
		\textcolor{magenta}{
			\underline{ \textcolor{black}{Exemple #1}} } }} {}
\newenvironment{exs}[1]{
	\subparagraph{ 	
		\textcolor{magenta}{
			\underline{ \textcolor{black}{Exemples #1}} } }} {}

\newenvironment{prop}[1]{
	\subparagraph{ 	
		\textcolor{cyan}{
			\underline{ \textcolor{black}{Proposition : #1}} } }} {}
\newenvironment{propriete}[1]{
	\subparagraph{ 	
		\textcolor{magenta}{
			\underline{ \textcolor{black}{Propriété : #1}} } }	} {}
