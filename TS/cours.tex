\documentclass[a4paper,10pt]{article}
\usepackage[utf8x]{inputenc}
\usepackage{amsmath}



\usepackage[pdftex]{graphicx}
\usepackage{color}


\newcommand{\Z}{\mathbb{Z}}
\newcommand{\N}{\mathbb{N}}
\newcommand{\R}{\mathbb{R}}

\newcommand{\HRule}{\rule[2mm]{0.4\linewidth}{0.5pt}}


\makeatletter
\newcommand\parag{%
   \@startsection{paragraph}{4}{0mm}%
      {-\baselineskip}%
      {.5\baselineskip}%
      {\normalfont\normalsize\bfseries}}


\makeatother




\newenvironment{definition}[1]
{	\parag{ \textcolor{green}{	\underline{ \textcolor{black}{Définition : #1}} } } \noindent}
{ \textcolor{green}{\HRule} \bigskip}

\newenvironment{thm}[1]{
	\parag{ 	
		\textcolor{red}{
			\underline{ \textcolor{black}{Théorème : #1}} } }  \noindent} 
			{  \textcolor{red}{\HRule}\bigskip }
			
\newenvironment{rem}[1]{
	\subparagraph{
		\textcolor{blue}{
			\underline{\textcolor{black}{Remarque #1}}}}} {}
			
\newenvironment{dem}[1]{
	\parag{
		\textcolor{yellow}{
			\underline{ \textcolor{black}{Démonstration : #1}} } }} {\textcolor{yellow}{\HRule} \bigskip}	
\newenvironment{lem}[1]{
	\subparagraph{ 	
		\textcolor{magenta}{
			\underline{ \textcolor{black}{Lemme : #1}} } } } {}		
\newenvironment{ex}[1]{
	\subparagraph{ 	
		\textcolor{magenta}{
			\underline{ \textcolor{black}{Exemple #1}} } }} {}
\newenvironment{exs}[1]{
	\subparagraph{ 	
		\textcolor{magenta}{
			\underline{ \textcolor{black}{Exemples #1}} } }} {}

\newenvironment{prop}[1]{
	\subparagraph{ 	
		\textcolor{cyan}{
			\underline{ \textcolor{black}{Proposition : #1}} } }} {}
\newenvironment{propriete}[1]{
	\subparagraph{ 	
		\textcolor{magenta}{
			\underline{ \textcolor{black}{Propriété : #1}} } }	} {}

%opening
\title{}
\author{}

\begin{document}

\maketitle

\begin{definition}{} Un processus aléatoire sur (\Omega, \tau, P) indéxé sur \R est une famille X_t , t \in \R de V. A. sur (\Omega, \tau, P).\\
Variante: indéxé sur \Z : (X_n)_n, n \in \Z
\end{definition}

Loi de X_t, t \in \R: c'est ka donnée des lois de tous les vecteurs.
Sauf cas particulier importants (Gauss, Markov), difficile à calculer et à manipuler

\paragraph{Notion de trajectoire} (correspond à la réalisation pour les variables aléatoires.\\
Vecteur aléatoire X : \Omega \rightarrow \R^{\N}
  \omega \mapsto X(w) : réalisation de la V.A. X

\subparagraph{Signal aléatoire :} \Omega \times T (domaine temporel) \R
(w,t) \mapsto \rightarrow X(w,t)
→ fonction de 2 variables

Si on fixe \omega : X_{\omega} : T \rightarrow \R
t \mapsto X_{\omega}(t) 
Trajectoire → revient à un signal déterministe.


\subparagraph{Description simplifiée} de X : \Omega \times T \to \R : ne retenir que des moments jusqu'à un certain ordre.

\begin{definition}{Moment d'ordre 1} On appelle moyenne du signal aléatoire X sur (\Omega, \tau, P) l'applications :\\
T \rightarrow \R
t  \mapsto m_X(t) = \E (X_t) 
\end{definition}
\begin{definition}{Moment d'ordre 2} On appelle covariance centrée \\
C_{X_c} : (t_1, t_2) \in T² → \E(C_{X_c}(t_1)C_{X_c}(t_2))\\
On appelle covariance (non centrée) 
E_X : (t_1, t_2) → E_X (t_1, t_2 ) = E(X(t_1)X(t_2))
\end{definition}

\begin{rem}
 VAR(X_t) = E((X_c(t))²) = C_{X_c}(t,t) : puissance moyenne du signal centré à l'instant t.
\rho_{X_{t_1}, X_{t_2}} = \dfrac{E(X_c(t_1)X_c(t_2))}{\sqrt(VAR(X_{t_1}) VAR(X_{t_2})}\\
      =	\dfrac{C_{X_c}(t_1, t_2)}{\sqrt{C_{X_c}(t_1, t_1)C_{X_c}(t_2, t_2)}}
\end{rem}

\begin{definition}{} X sur (\Omega, \tau, P) est stationnaire au sens strict si 
\forall N: \forall t_1, \dots, \t_N : loi de (X_{t_1 + \tau}, \dots, X_{t_N + \tau})^T ne dépend pas de \tau
\end{definition}

\begin{definition}{} 
On dit que moyenne m_X est stationnaire si m_X(t) ) cste (indep de t).
On dit que moyenne C_{X_c} est stationnaire si C_{X_c}(t_1 + \tau, t_2 + \tau) est indep de \tau (à t_1, t_2 fixés) ssi C_{X_c}(t_1, t_2) = f(t_1, t_2)
On dit que moyenne C_{X_c} est stationnaire si
X est stationnaire au sens lagre (SSL) si m_X stat et C_X stat.
\end{definition}


Ex : processus gaussien 
\forall N, \forall t_1, \dots, \t_N, 
\begin{matrix} X(t_1)\\ \dots \\ X(t_N) \end{matrix}
\sim \mathcal{N}\begin{matrix} m_X(t_1) \\ \dots \\ m_X(t_N) \end{matrix}, C_{[X_c(t_1), \dots, X_c(t_N)]^T} = 
particulièrement, si la loi de \begin{matrix} X(t_1)\\ \dots \\ X(t_N) \end{matrix} est absolument continue, on sait que 
p(x_1, \dots, x_N) = \frac{1}{2\Pi= (det[C


\begin{prop}
Si X gaussien, il suffit de connaître m_X(t), t \in \R et C_{X_c} (t_1, t_2) ,  (t_1, t_2) \in \R² pour connaître la loi.
\end{prop}

Exemples : 
Soit X(t) = A, A V.A. sur (\Omega, \tau, P).
X(\omega, t) = A(\omega) (indep de t)
m_X (t) = E(A) = m_A ) = cte
C_{X_c}(t_1, t_2) = E( X_c(t_1) X_c(t_2)) = E([A_c]²) = VAR(A)
\forall t_1, t_2 : \rho_{X(t_1)X(t_2)} = \rho_A = 1

X(\omega, t) = A(\omega)cos(2 \Pi \Phi_0 t + \phi(\omega)), t \in \R, \omega \in \Omega, \Phi_0 cste \not = 0
avec m_A, \sigma_A² connues
\phi et A indep
\phi de loi uniforme sue [0, 2\Pi]

\forall t, m_X(t) = E(Acos(2\Pi\Phi_0t + \phi))
 = E(A) E(cos(2\Pi\Phi_0t + \phi))
 = m_A E (f(\phi))
 = m_A \int f(u) p_{\phi} du = m_A \frac{1}{2\Pi}\int_0^{2\Pi}cos(2 \Pi \Phi_0 t + \phi)dt
 = 0
 

Remarque : si \phi est quelconque, on n'aura pas m_X(t) = 0

Covariance : C_{X_c}(t_1, t_2) = C_X(t_1, t_2) = E (A \cos(2\Pi f_0 t_1 + \phi)(A \cos(2\Pi f_0 t_2 + \phi)
 = E(A²) E(\cos(2\Pi f_0 t_1 + \phi) E(\cos(2\Pi f_0 t_2 + \phi) 
 = (m_A^2 + \sigma_A²) E( \frac{1}{2} [\cos(2\Pi f_0(t_1 + t_2) + 2 \phi) + \cos(2\Pi f_0 (t_1 - t_2))]
 = \frac{(m_A^2 + \sigma_A²) }{2} \cos(2\Pi f_0(t_1 + t_2)) + x \int_0^{2\Pi} \cos( \alpha + 2u)du

\Rightarrow C_X = C_{X_c} est stationnaire (fonction de t_1 - t_2)
On écrira : \Gamma_X(\tau) = \frac{(m_A^2 + \sigma_A²) }{2} \cos(2\Pi f_0 \tau), \tau \in \R : fonction d'auto-corrélation (statistique) de X.



Rappels (Processus Poisson)
Loi de Poisson \mathcal{P}(\lambda)
X \sim \mathcal{P}(\lambda) : 
P(X = k) = ...
Processus aléatoires ponctuel de poisson
\mathcal{N}_{(t_1, t_2)} = V.A. = nb d'évenements dans (t_1, t_2)
\sim \mathcal{P}(\lambda |t_1 - t_2 |)
(t_1, t_2), (t'_1, t'_2), \dots intervalles disjoints \rightarrow \mathcal{N}_{(t_1, t_2)}, \mathcal{N}_{(t_1, t_2)} mutuellement indépendants


Ex: Basculur Poissonien
(t_i) : instant d'un processus Poisson de densité constant \lambda \in \R^+
 P(X_0 = 1 ) = P(X_0 = 0 ) = \frac{1}{2}
\forall t, X_t à valeurs dans \{0, 1\}

t > 0,
E(X(t)) = 1 \times P(X(t) = 1) + 0 = P(X(t) = 1)
P(X(t) = 1) = P(X(t) = 1 / X(0) = 1) P(X(0) = 1) + P(X(t) = 1 / X(0) = 0) P(X(0) = 0)

C_X(t_1, t_2) = E(X(t_1), X(t_2)) = 1 \times P(X(t_1) = 1 , X(t_2) = 1) 
  = P(X(t_2) = 1 / X(t_1) = 1) P(X(t_1) = 1)
 = P( N_{]t_1; t_2]} pair) \times \frac{1}{2} 

N_{]t_1; t_2]} \sim \mathcal{P}(\lambda(t_2 - t_1) \Rightarrow (N_{]t_1; t_2]} = \sum_{n=0}^{\infty} e^{-\lambda \tau} \frac{(\lambda \tau)^{2n}}{2n!} = e^{-\lambda \tau} \sum_{n=0}^{\infty} \frac{(\lambda \tau)^{2n}}{2n!} = e^{-\lambda \tau} ch(\lambda \tau), \tau = |t_1 - t_2|


C_X(t_1, t_2) = \frac{1}{2} \times e^{-\lambda \tau} \frac{e^{\lambda \tau} + e^{-\lambda \tau}}{2}
 = \frac{1}{4} [1 + e^{-2\lambda \tau}], \tau \in \R

Propriété  : C_X(t_1,t_2 )  = C_{X_c}(t_1,t_2 ) + m_X(t_1)m_X(t_2), t_1, t_2 \in \R

\end{document}
