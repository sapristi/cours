\documentclass[a4paper,10pt]{article}
\usepackage[utf8x]{inputenc}
\usepackage{amsmath}



\usepackage[pdftex]{graphicx}
\usepackage{color}


\newcommand{\Z}{\mathbb{Z}}
\newcommand{\N}{\mathbb{N}}
\newcommand{\R}{\mathbb{R}}

\newcommand{\HRule}{\rule[2mm]{0.4\linewidth}{0.5pt}}


\makeatletter
\newcommand\parag{%
   \@startsection{paragraph}{4}{0mm}%
      {-\baselineskip}%
      {.5\baselineskip}%
      {\normalfont\normalsize\bfseries}}


\makeatother




\newenvironment{definition}[1]
{	\parag{ \textcolor{green}{	\underline{ \textcolor{black}{Définition : #1}} } } \noindent}
{ \textcolor{green}{\HRule} \bigskip}

\newenvironment{thm}[1]{
	\parag{ 	
		\textcolor{red}{
			\underline{ \textcolor{black}{Théorème : #1}} } }  \noindent} 
			{  \textcolor{red}{\HRule}\bigskip }
			
\newenvironment{rem}[1]{
	\subparagraph{
		\textcolor{blue}{
			\underline{\textcolor{black}{Remarque #1}}}}} {}
			
\newenvironment{dem}[1]{
	\parag{
		\textcolor{yellow}{
			\underline{ \textcolor{black}{Démonstration : #1}} } }} {\textcolor{yellow}{\HRule} \bigskip}	
\newenvironment{lem}[1]{
	\subparagraph{ 	
		\textcolor{magenta}{
			\underline{ \textcolor{black}{Lemme : #1}} } } } {}		
\newenvironment{ex}[1]{
	\subparagraph{ 	
		\textcolor{magenta}{
			\underline{ \textcolor{black}{Exemple #1}} } }} {}
\newenvironment{exs}[1]{
	\subparagraph{ 	
		\textcolor{magenta}{
			\underline{ \textcolor{black}{Exemples #1}} } }} {}

\newenvironment{prop}[1]{
	\subparagraph{ 	
		\textcolor{cyan}{
			\underline{ \textcolor{black}{Proposition : #1}} } }} {}
\newenvironment{propriete}[1]{
	\subparagraph{ 	
		\textcolor{magenta}{
			\underline{ \textcolor{black}{Propriété : #1}} } }	} {}

%opening
\title{}
\author{}

\begin{document}

\maketitle

\begin{definition}{} Un processus aléatoire sur (\Omega, \tau, P) indéxé sur \R est une famille X_t , t \in \R de V. A. sur (\Omega, \tau, P).\\
Variante: indéxé sur \Z : (X_n)_n, n \in \Z
\end{definition}

Loi de X_t, t \in \R: c'est ka donnée des lois de tous les vecteurs.
Sauf cas particulier importants (Gauss, Markov), difficile à calculer et à manipuler

\paragraph{Notion de trajectoire} (correspond à la réalisation pour les variables aléatoires.\\
Vecteur aléatoire X : \Omega \rightarrow \R^{\N}
  \omega \mapsto X(w) : réalisation de la V.A. X

\subparagraph{Signal aléatoire :} \Omega \times T (domaine temporel) \R
(w,t) \mapsto \rightarrow X(w,t)
→ fonction de 2 variables

Si on fixe \omega : X_{\omega} : T \rightarrow \R
t \mapsto X_{\omega}(t) 
Trajectoire → revient à un signal déterministe.


\subparagraph{Description simplifiée} de X : \Omega \times T \to \R : ne retenir que des moments jusqu'à un certain ordre.

\begin{definition}{Moment d'ordre 1} On appelle moyenne du signal aléatoire X sur (\Omega, \tau, P) l'applications :\\
T \rightarrow \R
t  \mapsto m_X(t) = \E (X_t) 
\end{definition}
\begin{definition}{Moment d'ordre 2} On appelle covariance centrée \\
C_{X_c} : (t_1, t_2) \in T² → \E(C_{X_c}(t_1)C_{X_c}(t_2))\\
On appelle covariance (non centrée) 
E_X : (t_1, t_2) → E_X (t_1, t_2 ) = E(X(t_1)X(t_2))
\end{definition}

\begin{rem}
 VAR(X_t) = E((X_c(t))²) = C_{X_c}(t,t) : puissance moyenne du signal centré à l'instant t.
\rho_{X_{t_1}, X_{t_2}} = \dfrac{E(X_c(t_1)X_c(t_2))}{\sqrt(VAR(X_{t_1}) VAR(X_{t_2})}
      =	

\end{rem}



\end{document}
