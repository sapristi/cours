\documentclass[10pt,a4paper]{article}
\usepackage[utf8x]{inputenc}
\usepackage{ucs}
\usepackage{amsmath}
\usepackage{amsfonts}
\usepackage{amssymb}
\usepackage{hyperref}
\usepackage{pstricks}

\title{Conception et vérification de programmes}


\usepackage[pdftex]{graphicx}
\usepackage{color}


\newcommand{\Z}{\mathbb{Z}}
\newcommand{\N}{\mathbb{N}}
\newcommand{\R}{\mathbb{R}}

\newcommand{\HRule}{\rule[2mm]{0.4\linewidth}{0.5pt}}


\makeatletter
\newcommand\parag{%
   \@startsection{paragraph}{4}{0mm}%
      {-\baselineskip}%
      {.5\baselineskip}%
      {\normalfont\normalsize\bfseries}}


\makeatother




\newenvironment{definition}[1]
{	\parag{ \textcolor{green}{	\underline{ \textcolor{black}{Définition : #1}} } } \noindent}
{ \textcolor{green}{\HRule} \bigskip}

\newenvironment{thm}[1]{
	\parag{ 	
		\textcolor{red}{
			\underline{ \textcolor{black}{Théorème : #1}} } }  \noindent} 
			{  \textcolor{red}{\HRule}\bigskip }
			
\newenvironment{rem}[1]{
	\subparagraph{
		\textcolor{blue}{
			\underline{\textcolor{black}{Remarque #1}}}}} {}
			
\newenvironment{dem}[1]{
	\parag{
		\textcolor{yellow}{
			\underline{ \textcolor{black}{Démonstration : #1}} } }} {\textcolor{yellow}{\HRule} \bigskip}	
\newenvironment{lem}[1]{
	\subparagraph{ 	
		\textcolor{magenta}{
			\underline{ \textcolor{black}{Lemme : #1}} } } } {}		
\newenvironment{ex}[1]{
	\subparagraph{ 	
		\textcolor{magenta}{
			\underline{ \textcolor{black}{Exemple #1}} } }} {}
\newenvironment{exs}[1]{
	\subparagraph{ 	
		\textcolor{magenta}{
			\underline{ \textcolor{black}{Exemples #1}} } }} {}

\newenvironment{prop}[1]{
	\subparagraph{ 	
		\textcolor{cyan}{
			\underline{ \textcolor{black}{Proposition : #1}} } }} {}
\newenvironment{propriete}[1]{
	\subparagraph{ 	
		\textcolor{magenta}{
			\underline{ \textcolor{black}{Propriété : #1}} } }	} {}

\usepackage[amsmath,thmmarks,framed]{ntheorem2}
\usepackage{color}
\usepackage{framed}

%\usepackage[pdftex]{graphicx}
\usepackage{graphicx}

\newcommand{\Z}{\mathbb{Z}}
\newcommand{\N}{\mathbb{N}}
\newcommand{\R}{\mathbb{R}}

\newcommand{\HRule}{\rule[0mm]{0.4\linewidth}{0.5pt}}


%\newcommand{\HRule}{\rule[2mm]{0.4\linewidth}{0.5pt}}

\makeatletter
\newtheoremstyle{nonumberplainperso}
  {\item[\hskip\labelsep \theorem@headerfont ##1\  ]}%
  {\item[\hskip\labelsep \theorem@headerfont ##1\  : ##3]}
\newtheoremstyle{nonumberbreakperso}
  {\item[\rlap{\vbox{\hbox{\hskip\labelsep \theorem@headerfont
          ##1 \theorem@separator}\hbox{\strut}}}]}%
  {\item[\rlap{\vbox{\hbox{\hskip\labelsep \theorem@headerfont
          ##1\ : ##3\theorem@separator}\hbox{\strut}}}]}
\makeatother


\newcommand{\defcolor}{green}
\newcommand{\thmcolor}{red}
\newcommand{\rmcolor}{blue}
\newcommand{\demcolor}{yellow}
\newcommand{\lemcolor}{magenta}
\newcommand{\propcolor}{cyan}
\newcommand{\currentcolor}{\defcolor}

\renewcommand*\FrameCommand{{\color{\currentcolor}\vrule width 3pt  \HRule \hspace{20pt}\hspace{-0.4\linewidth}  }}


\theoremstyle{nonumberbreakperso}
\theoremseparator{.}
\theoremprework{\bigskip\hrule}
\theorempostwork{\hrule\bigskip}
\newframedtheorem{adefinition}{\textcolor{\currentcolor}{\underline{ \textcolor{black}{Définition}}}}


\theoremstyle{nonumberbreakperso}
\theoremheaderfont{\normalfont\bfseries}\theorembodyfont{\slshape}
\theoremseparator{}
\newframedtheorem{athm}{\textcolor{\currentcolor}{\underline{ \textcolor{black}{Théorème}}}}

\theoremstyle{nonumberplainperso}
\theorembodyfont{\upshape}
\theoremseparator{}
\newframedtheorem{arem}{ \textcolor{\currentcolor}{\underline{ \textcolor{black}{Remarque}}}} 
\newframedtheorem{aex}{ \textcolor{\currentcolor}{\underline{ \textcolor{black}{Exemple}}}}
\newframedtheorem{aexs}{ \textcolor{\currentcolor}{\underline{ \textcolor{black}{Exemples}}}}


\theoremheaderfont{\sc}\theorembodyfont{\upshape}
\theoremstyle{nonumberplainperso}
\theoremseparator{}
\theoremsymbol{\rule{1ex}{1ex}}

\newframedtheorem{adem}{ \textcolor{\currentcolor}{\underline{ \textcolor{black}{Dém}}}}

\theoremstyle{nonumberbreakperso}
\theoremindent0.5cm
\newframedtheorem{alem}{ \textcolor{\currentcolor}{\underline{ \textcolor{black}{Lemme}}}}

\theoremindent0cm

\newframedtheorem{aprop}{ \textcolor{\currentcolor}{\underline{ \textcolor{black}{Proposition}}}}
\newframedtheorem{apropriete}{ \textcolor{\currentcolor}{\underline{ \textcolor{black}{Propriété}}}}


\newenvironment{definition}[1][]{\renewcommand{\currentcolor}{\defcolor}\begin{adefinition}[#1]}{\end{adefinition}}
\newenvironment{thm}[1][]{\renewcommand{\currentcolor}{\thmcolor}\begin{athm}[#1]}{\end{athm}}
\newenvironment{rem}[1][]{\renewcommand{\currentcolor}{\rmcolor}\begin{arem}[#1]}{\end{arem}}
\newenvironment{ex}[1][]{\renewcommand{\currentcolor}{\rmcolor}\begin{aex}[#1]}{\end{aex}}
\newenvironment{exs}[1][]{\renewcommand{\currentcolor}{\rmcolor}\begin{aexs}[#1]}{\end{aexs}}
\newenvironment{dem}[1][]{\renewcommand{\currentcolor}{\demcolor}\begin{adem}[#1]}{\end{adem}}
\newenvironment{lem}[1][]{\renewcommand{\currentcolor}{\lemcolor}\begin{alem}[#1]}{\end{alem}}
\newenvironment{prop}[1][]{\renewcommand{\currentcolor}{\propcolor}\begin{aprop}[#1]}{\end{aprop}}
\newenvironment{propriete}[1][]{\renewcommand{\currentcolor}{\propcolor}\begin{apropriete}[#1]}{\end{apropriete}}

\newcommand{\F}{\mathcal{F}}
\newcommand{\ssi}{$ ssi $}

\begin{document}


\url{inria.fr/prive/fschwarz/cvfp}


\section{Machines de Turing}

\begin{definition}[Machine de Turing] $M = ( \Sigma , Q , \delta, s)$ tel que \begin{itemize}
\item $\Sigma$ : alphabet fini
\item Q : ensemble d'états fini non vide
\item $\delta \subseteq (Q \times \Sigma) \times (Q \cup \{yes \} \times \Sigma \times \{-1, 0, 1\}$
\item $s \in Q$
\end{itemize}
On dit que M est déterministe si $\forall q \in Q, \forall a \in \Sigma,\\
 \mid \{ (q', b, d) \mid (q, a, q', b, d) \in \delta \} \mid \leq 1$
\end{definition}

\subsection{Executions}
\begin{definition}{Configuration}
Une configuration est un triplet $(q, k, w)$ où \begin{itemize}
\item $q$ est un état
\item $k \in \N$
\item $w$ est un mot fini sur $\Sigma$
\end{itemize}
\end{definition}

\begin{definition}{Transition}
$(q, k, w) \rightarrow (q', k', w')$ ssi $(q, \sigma, q', \sigma', dir) \in \delta$ et \begin{itemize}
\item $w[k] = \sigma$
\item $w' = w$ sauf $w'[k] = \sigma'$
\item $k' = k + dir$
\item $k' \geq 0$
\end{itemize}\end{definition}

\begin{definition}{} Soient c, c' deux configurations, \begin{itemize}
\item $c \rightarrow^0 c$
\item $ \forall t \geq 1, c \rightarrow^t c' \ssi \exists c''$ configuration, $c \rightarrow^{t-1}c''$ et $c'' \rightarrow c'$
\item $c \rightarrow^* c'$ ssi $\exists t \geq 0 / c \rightarrow^* c'$
\end{itemize} \end{definition}

\subsection{Décision} 

 \begin{definition}{}
M décide un language $L \subseteq \Sigma^*$ ssi $\forall x \in \Sigma^*, x \in L ssi \exists k \geq 0, w \in \Sigma^* / (s, 0, x) \rightarrow^* (yes, k, w)$\\
  \end{definition}

 \begin{definition}{}
 Soit $f : \N \rightarrow \N$, M décide L en temps f ssi M décide L et $\forall t \in \N, \forall x in \sigma^*, (s, 0, x) \rightarrow ^t (q, k, w)$ alors $t \leq f(|x|)$\\
 \end{definition}

\begin{prop}{}
 Soit M qui décide L en temps f, si $(s, 0, x \rightarrow^* (q, k, w)$, alors $|w| \leq max ( |x|, f(|x|)$
\end{prop}
 \begin{dem}{}
  Au temps t, $\forall k' > max ( |x|, t), w[k'] )$ est vide et le curseur courant $k \leq t$\\
 \end{dem}

 \part{Complexité en temps}
 
 \begin{definition}{}\begin{itemize}
 \item P est la classe des langages $L / \exists f$, polynôme, M machine det / M décide L en temps f
\item NP M non det
\item EXPTIME f polynome, M machine det, M décide L en temps $2^f$
 \item NEXPTIME M non det
 \end{itemize}
 \end{definition}


 \subsection{Difficultés}
 \begin{definition}{}L' réductible en temps polynomial à L ssi $\exists f$, calculable en temps polynomial $/ \exists x \in \Sigma ^*, x \in L' \Leftrightarrow f(x) \in L$
 \end{definition}
 \begin{definition}{}L est NP dur ssi pour tout $L' \in NP$, on a L' réductble en temps polyomial à L
 Lest NP complet ssi $L \in NP$ et L est NP hard
 \end{definition}
 Rem Si L est NP hard, $L \in P$ alors NP = P
 
 \subparagraph*{Prop} L est NP dur, L réductible en temps poly à L', alors L' est NP dur.
 L' in NP, L réductible à $L' \Leftrightarrow L \in NO$
 
 \begin{definition}{}CONP est l'ensemble des $L / \Sigma^* L \in NP$
 \end{definition}
 \part{Pavages et logique proportionnelle}
 
 \subsection{Pavages} Un tyoe de pavé t est un quadruplet $< left(t), top(t), right(t), down(t) > \in \N ^4$
 TILING est le problème suivant : \begin{itemize}
 \item entrée : T esnsemble fini de types de pavé $t_1, ..., t_n \in T$
\item sortie : oui si $\exists \mu : \{ 1, ..., n\} ^2 \rightarrow T /$ \begin{itemize}
\item $\exists j \in \{1, .., n\}, \mu(1,j) = t_j$
\item $\forall i < n, \forall j < n, down(\mu(i,j)) = up(\mu(i+1, j))$
\item $\forall i \leq n , \forall j leq right mu (i,j) = leftmu (i, j+1))$
\end{itemize}
\end{itemize}

Theoreme : TILING est NP hard

\subparagraph*{Théorème} SAT est NP-hard

Dem : On transforme une instance du pavage en une formule $\Phi$ de la logique prop. en encodant les contraintes avec les variables $p_{i,j,t}$ pour $\mu(i,j) = t$.



\subsection{Complexité en espace}
\begin{definition}{}L est dans PSPACE ssi $\exists M$ machine déterministe, f un polynome tels que \begin{itemize}
\item M décide L
\item $\forall x in \Sigma^x, (s,0,x) \rightarrow^* (q, k, w)$, on a $|w| \leq f(|x|)$
\end{itemize}\end{definition}
\begin{definition}{}L est dans NPSPACE ssi $\exists M$ machine non déterministe, f un polynome tels que \begin{itemize}
\item M décide L
\item $\forall x \in \Sigma^x, (s,0,x) \rightarrow^* (q, k, w), on a |w| \leq f(|x|)$
\end{itemize}\end{definition}
\begin{definition}{}L est dans EXPSPACE ssi $\exists M$ machine déterministe, f un polynome tels que \begin{itemize}
\item M décide L
\item $\forall x in \Sigma^x, (s,0,x) \rightarrow^* (q, k, w)$, on a $|w| \leq 2^{f(|x|)}$
\end{itemize}\end{definition}
\begin{definition}{}L est dans NEXPSPACE ssi $\exists M$ machine non déterministe, f un polynome tels que \begin{itemize}
\item M décide L
\item $\forall x in \Sigma^x, (s,0,x) \rightarrow^* (q, k, w), on a |w| \leq 2^{f(|x|)}$
\end{itemize}\end{definition}
\subparagraph*{Rem} Normalement, les classes de complexité en espace se définissent avec une machine de Turing à plusieurs rubans

\subsection{Un autre pavage}
CORRIDOR TILING est défini par :
\begin{itemize}
\item entrée : T ensemble fini de type de pavé, et $t_1, ..., t_n, b_1, ..., b_n$ deux séquences de types de pavés
\item sortie : oui si on peut paver 
$\begin{array}{c|c|c|c|c|c}
t_1 & & & & & t_n \hbar \\
& & & & & \\
& & & & & \\
& & & & & \\
& & & & & \\
b_1 & & & & & b_n
\end{array}$ n fois $2^n$
\end{itemize}

\begin{definition}{}$L \in PSAPCE-hard$ ssi pour tout L' dans PSAPCE, L' est réductible en temps polynomial à L
\end{definition}
\subparagraph*{Theroreme} CORRIDOR TILING est NPSAPCE-hard
Dem : Soit L dans NPSPACE, il existe M machie non déterministe et f un polynome, tels que \begin{itemize}
\item M décide L
\item M utilise un espace mémoire en f
\end{itemize}
On suppose de plus que la machine efface son ruban si elle accepte x et arrive sur $(yes, 0, \varepsilon)$\\
Considérons les executions possibles de M sur |x|. Le nombre de configs possibles est au plus $|Q| \times f(|x|) \times  (1 + |\Sigma|)^{f|x|)}$\\
Si M accepte x, il existe alors une exécution de longueur au plus $c^{f(|x|)}$ où c est une constante. On peut donc encoder cette exécution dans un pavage : 
$\begin{array}{c||c|c|c|c|c|c||c |c}
& t_1 & & & & & t_n & $encode la config init$ \hbar \\
& & & & & & & & \\
& & & & & & & & 2 ^{log c f(|x|)} = c^{f|x|} \\
& & &&  & & & & \\
& & & & & & & & \hbar\\
& & & & & & & & encode (yes, 0, \varepsilon)
\end{array}$
$\log_2 c f|x|$

\subsection{Logique proportionnelle quantifiée}
SAT : est ce qu'il existe une valuation $\mu$ telle que $\mu \models \Phi$ ?
VALID : est ce que pour toute valuation $\mu$ on a $\mu \models \Phi$ ?

\subparagraph*{def} Une ormule de la logique prop qiantifiée (QBF, quantifiable binary formulaà est une formule de la forme ; 
il $\exists p1, \forall p2, \exists p3 ... Q_np_n \Phi$ où
$Q_n \in { \exists, \forall }$
Phi est formule de la logique propositionnelle.
\begin{definition}{}On définit $\mu \models \Phi$ pour $\Phi$ de la logique prop quantfiée \begin{itemize}
\item $\mu \models \exists p \Phi$ ssi $\mu [ p := 0] \models$ ou $\mu [p := 1] \models \Phi$
\item $\mu \models \forall p \Phi$ ssi $\mu [ p := 0] \models$ et $\mu [p := 1] \models \Phi$
\item $\mu \models  \Phi$ habituel pour $\Phi$ propositionnel
\end{itemize}\end{definition}
\subparagraph*{Theoreme} Le probleme QBF-sat / entrée : 
$\Phi \exists p_1 \forall p_2 Q_n p_n \Psi$ où p1 p2 pn sont les variables qui apparaisent dans $\Psi$
sortie : oui si $\Phi$ est vrai(ie $\mu \models \Phi, \mu$ quelconque)
est NPSAPCE hard

\subparagraph*{dem} 
but:  transormer un instance de corridor tiling en $\Phi$ de la logique prop quantifiée / ..; pavable si $\Phi$ est vraie et de plus Phidoit etre de taille polynomiale par rapport à l'instance -> diviser pour régner

On note $\vec{p_a} = (p_a, j, t)_{j \in { 1, ..; n}, t \in T}$ Cela permet de décrire une ligne du pavage.

1ere idée : Définir $\Phi_k (p_a, p_b)$ qui encode le fait qu'il existe un pavage $n \times 2^k , p_a$ 1ere ligne, $p_b$ dernière ligne.
On définit $\Phi_1 (p_a, p_b) = \dots$
Par récurence, $\Phi_k(p_a, p_b) = \exists p_c, p_d / phi_k-1 (p_a, p_c)$ et $ph_1(pc, pd)$ et $\Phi_k-1(p_d, p_b) \forall k \geq 2$
Problème : $\Phi_n$ va être de taille exponentielle

2eme idée : (la bonne)
$\Phi_1 (p_a, p_b) = ...$
$\Phi_k(p_a, p_b) = \exists p_c, p_d / ph_1(pc, pd)$ et $\forall p_q, p_r , (((p_q \Leftrightarrow p_a) et (p_r \Leftrightarrow p_c)$ ou $(p_q \Leftrightarrow p_d) et (p_r \Leftrightarrow p_b))$ implique $\Phi_k-1 (p_q, p_r)$


Par exemple $p_q \Leftrightarrow p_a$ est un raccourci pour $\bigwedge_{j=1}^n \bigwedge t \in T (p_{q,j,t} equiv p_{a, j, t}$

\subparagraph*{Th} $QBF sat \leqslant est PSPACE$
Dem : thèse de church\\
fonction qbf $sat(\mu, phi)$
	cas  (Phi)
		$\Phi = \exists p \Psi : qbf-sat(\mu [p:=0], \Psi)$ ou $qbf-sat (\mu[p:=1] \Psi)$
		$\Phi = \forall p \Psi, qbf-sat(\mu [p:=0], \Psi)$ et $qbf-sat (\mu[p:=1] \Psi)$
		$\Phi$ propositionnel : retourne oui ssi $\mu \models Psi$
qbf-sat s'execute en espace polynomial


\subsection{Formule de Sahlquist}
On définit une construction syntaxique 
$S ::= X \rightarrow POS$
où POS est une formule positive et $\chi ::= T | \perp | \Box_{i_1} \cdots \Box_{i_n}p | \chi \wedge \chi | \Diamond_{j_i} \chi$
La formule générée par S
\begin{thm}{} Soit $\Phi$ une formule de Sahlquist. Il existe une formule du 1er ordre $\alpha$ qui correspond à $\Phi$ sur les cadres.\\
Pour tout cadre $\F$, pour tout monde $m$,
$\F, m \models \Phi$ ssi $\F[x \leftarrow m] \models \alpha(x)$
\end{thm}
\begin{ex}{1}
$ \F, m \models \Phi$ ssi $\forall V, \F, V, m \models \Phi$
ssi $\forall V, \F, V, w \models p$ implique $\F, V, w \models \Diamond p$
ssi $\forall V, w \in V(p)$ implique $\F, V, m \models \Diamond p$
ssi $\forall V' / V'(p) = {w}$ implique $\F, V, w \models p$
ssi $\forall V' / V'(p) = \{ w \}$ implique $\F, V, w \models \Diamond p$
(car $\Diamond p$ est croissante en p car positive)
(1) ssi $\forall V' / V'(p) = w$ on a $\F, V', [x \leftarrow w ] \models \exists y xRy \wedge p(y)$
ssi le prédicat p est interprété avec la fonction 
$\begin{array}{ll}
 W \longleftarrow \{0, 1\}\\
 w \mapsto & 1 si v = w\\
	    & 0 sinon
\end{array}$
On a montré que $\forall w \in W$
$\F, w \models p \to \Diamond p ssi F[x \leftarrow w] \models x Rx$

On a aussi:
$\F \models p \to \Diamond p ssi \forall w, \F[x←w] \models xRx
			    ssi \F \models \forall x, xRx$
\end{ex}

1)Traduire $\chi \leftarrow POS$ en second ordre de la forme
$\forall \vec p, \forall \vec y, [(REL \wedge BOX) \leftarrow ST_x(POS)$
où RES est une conjoniotn de $y_i R_k y_j x R_k y_i$
   BOX est une conjonction de la forme $\forall s, y_i R_{\beta} s \leftarrow p(s)$
2) Retourne la formule $\forall \vec y, REL \leftarrow ST_x(POS)$ dans laquelle on a remplacé les occurences de p par 
$\sigma(p)$ par $\lambda a . y_{i_1} R_{\beta_1} a \vee \cdots \vee y_{i_k} R_{\beta_k} a$\\
où chaque $y_{i_j} R_{\beta_j} a$ provient d'une formule $\forall s y_{i_j} R_{\beta_j} s \leftarrow p(s)$ qui apparaît dans BOX.

\subsection{Théorème de complétude}

Soit S un ensemble de formules . On écrit $\vdash_S \Phi$ pour dire que $\Phi$ est prouvable en utilisant 
\begin{itemize}
 \item les tautologies
 \item l'axiome K
 \item les instances de formules de S
 \item le modus ponens
 \item la nécéssitation
\end{itemize}
Rem : $\vdash \Phi; \vdash_{\empty} \Phi$
Soit $\mathcal{C}$ une classe de cadres.
$\models_{\mathcal{C}} \Phi$ ssi $\forall \F \in \mathcal{C}, \F \models \Phi$
$\Sigma \models_{\mathcal{C}} \Phi$ ssi pour tout modèle $\mathcal{M}, w$ basé sur un cadre de $\mathcal{C}$, on a :
Si $\mathcal{M}, w \models \Sigma alors \mathcal{M}, w \models \Phi$

\begin{thm}{de complétude de Sahlquist}
Soit S un ensemble de formules de Sahlquist, soit $C_s$ la classe de cadres qui vérifient les propriétés du 1er ordre correspondantes aux formules de S sur les cadres.
Alors  $\Sigma \vdash_S \Phi$ ssi $\Sigma \models_{\mathcal{C}} \Phi$ (complétude forte)
(complétude faible : $\vdash_S \Phi$ ssi $\models_S \Phi$.)
\end{thm}

Trois techniques pour montrer la décidabilité/compléxité du problème de logique modale.

\subsection{Logique modale S5}
\begin{tabular}{c | c  }
 axiomatique & propriété sur les cadres \\
 $\Box p \leftarrow p$ & réfléxivité\\
 $\Box p \leftarrow \Box \Box p$ & transitivité\\
 $\lnot\Box p \Box \lnot \Box p$ & euclidianité\\
 $(\Diamond \Box p \leftarrow \Box p)$\\
\end{tabular}
\subparagraph{Application : } Logique epistémique

euclidianité : $\forall x, y, z, (x R y \wedge xRz) \leftarrow y R z$

\begin{prop}
 Si R est réfléxive, transitive, alors R euclidienne ssi R symétrique
\end{prop}
\begin{prop}
Soit $\Phi$ satisfiable dans un modèle où R est une rel d'équivalence 
$\Leftrightarrow \Phi$ satisfiable dans un modèle M = (W, R, V) où $R = W \times W$
\end{prop}
Dem \fbox{$\Leftarrow$} : trivial
\fbox{$\Rightarrow$} Si il existe M, w où M  = (W, R, V) tq
\begin{itemize}
 \item R est une relation d'équivalence
 \item $M, w \models \Phi$
\end{itemize}





\subsection{Méthodes de tableau}

KT = logique des cadres réflexifs.
S5 = logique des cadres où R est une relation d'équivalence.
S4 = logique des cadres où R est réflexive et transitive.

On écrit les formules sous forme normale négative avec la logique suivante :\\
 $ \Phi := p | \lnot p | \perp | \top | \Phi \vee \Phi | \Phi \wedge \Phi | \Diamond \Phi | \Box \Phi $

\begin{definition}
 Soit $\Sigma$ un sous-ensemble de formules en FNN clos par sous-formules. Un ensemble de S4-Hintickka H sur $\Sigma$ est un ensemble qui satisfait:
\begin{itemize}
 \item $H \subseteq \Sigma$
 \item $\perp \not \in H$
 \item $p \in H$ ssi $\lnot p \not \in H$
 \item Si $\Phi\vee\Psi \in H,$ alors $\Phi \in H$ ou $\Psi \in H$
  \item Si $\Phi\wedge\Psi \in H,$ alors $\Phi \in$ H et $\Psi \in H$
  \item Si $\Box \Phi \in H,$ alors $\Phi \in H$
\end{itemize}
On note $H^{\Box} = \{\Box \Psi \in H \}$
\end{definition}


Idée : Pour tester si $\Phi$ est S4-satisfiable, on appelle satS4$(\{\Phi\}, [])$
$\Phi = $

\begin{thm}{}
 L'appel S4$(\{\Phi\}, [\,])$ termine et utilise un espace polynomial en la taille de $\Phi$
\end{thm}
\begin{dem}
 Tous les $\Gamma$ des différents appels sont de la forme $\{\Psi\} \cup \Box)$
Invariant à chaque appel : 
\begin{itemize}
 \item Si $L = [ \Gamma_1, \cdots, \Gamma_k]$ alors $\Gamma_1^{\Box} \subseteq \cdots \subseteq \Gamma_k^{\Box} \subseteq \Gamma^{\Box}$
 \item chaque $\Gamma_i$ est de la forme $\Gamma_i^{\Box} \cup \{\Psi_i\}$
 \item Les éléments de L sont distincts
 \item Tous les ensembles ($\Gamma$, les $\Gamma_i$ sont des sous-ensembles de $SF(\Phi)$
\end{itemize}

But : $|L| \leq |\Phi|^2 + |\Phi| +1$
$\Gamma_1^{\Box} \subseteq \cdots \subseteq \Gamma_k^{\Box} \subseteq  SF(\Phi)$
$SF(\Phi)$ est de cardinal $|\Phi|$ : Il y a au plus $|\Phi|$ inclusions strictes.

Soient $i_1, i_2, \dots$ les indices tq $\Gamma_{i_j-1}^{\Box} \varsubsetneq \Gamma_{i_j}^{\Box}$
On a $i_{j+1} - i_j \leq |\Phi|$ car sinon il y aurait deux fois la même formule $\Psi_i$ et donc 2 éléments de la liste L seraient égaux.

$k = \sum^J_{j=0} i_{j+1} - i_j \leq |\Phi|^2$
\end{dem}

\begin{thm}{}
 Si $\Phi$ est satisfiable, alors satS4$(\{\Phi\}, [])$ réussit
\end{thm}
\begin{dem}{}
 Par récurrence sur $|\Phi|^2 - L$, on montre que le sous-appel satS4$(\Gamma, L)$ réussit :
Si $\Gamma \in L$, l'appel réussit
Sinon,$H = \{\Psi \in SF(\Phi) / M, w \models \Psi \}$ et on considère l'exécution qui s'occupe de ce H là.
Pour tout $\Diamond \Psi \in H, \{ \Psi \} \cup H^{\Box}$ est sat en $M,u$.
$\Rightarrow$ Par récurrence, les sous appels réussissent $\Rightarrow$ Cet appel réussit
\end{dem}

\begin{thm}{}
 Si satS4$(\{\Phi\}, [])$ réussit, alors $\Phi$ satisfiable.
\end{thm}

\begin{dem}{}
 \fbox{1} On montre que si  satS4$(\Gamma, L)$ réussit, alors il existe une structure $\mathcal{S} = (S, T, F)$ /
\begin{itemize}
 \item $(S,T)$ est un ordre
 \item F etiquette les noeuds de l'ordre avec des ensembles S4-Hintikka, et les feuilles sont étiquetées par des ensembles $\Gamma$ tels que
\begin{itemize}
 \item soit $\Gamma$ ne contient pas de $\Diamond \Psi_i$
 \item soit $\Gamma \in L$
 \item soit $\Gamma$ est inclus dans l'étiquette d'un label d'un prédécesseur
\end{itemize}
\item Si $\Box \Psi \in F(s)$, alors $t \in T(s), \Box \Psi \in F(t)$
\item Si $\Diamond \Psi \in F(s)$, alors $\exists t \in T(s),  \Psi \in F(t)$
\item La racine est étiquetée par un S4-hintikka incluant $\Gamma$
\end{itemize}

\fbox{2} On transforme $\mathcal{S} = (S, T, F)$ qui correspond à satS4$(\Gamma, L)$ en un modèle de Kripke.
Soit $\mathcal{M} = (W, R^*, V)$ défini par 
\begin{itemize}
 \item $W = S I$ où $I$ est l'ensemble des feuilles incluses dans un prédécesseur
 \item $R = T_{|_W} \cup \{ (w,u) \in W / \exists l \in I / (w,l) \in T et F(l) \subseteq F(u)$
 \item $V(p) =  \{w \in W, / p \in F(w) \}$
\end{itemize}
Et $R^*$ : fermeture réflexive et transitive de R

On montre par induction sur $\Phi \in SF(\Phi)$ qui $\forall w \in W$, 
$\Psi \in F(w) ssi M,w \models \Psi$
\end{dem}

\begin{thm}{}
 S4-SAT est dans PSPACE
\end{thm}
\begin{thm}{}
 S4-SAT est PSPACE-hard
\end{thm}

\subsection{Filtration} 
But : Si $M, w \models \Phi$, trouver $M^f, u / $
\begin{itemize}
 \item $M^f$ fini
 \item $M^f, u \models \Phi$
\end{itemize}

\begin{definition}{}
 Soit $\Phi$ une formule. Soit $M = (W, R, V)$. On définit $w \leftrightsquigarrow u$ ssi $\forall \Psi \in SF(\Phi),$
$M, w \models \Psi$ ssi $M, u \models \Phi$
On dit que $M^f = (W^f, R^f, V^f)$ est une filtration si 
\begin{enumerate}
 \item $W = W_{| \leftrightsquigarrow}$
 \item Si $w R v$ alors $|w| R^f |v|$
 \item Si $|w| R^f |v|$ alors $\forall \Box \Psi \in SF(\Phi), $
	si M, w $\models \Box Phi$, alors $M,u \models \Psi$
\item $V^f(p) \{ [ w | in W^f / M,w models p \}$ pour $p$ qui apparait dans $\Phi$
\end{enumerate}

\end{definition}

\begin{prop}{}
$|W^f| leq 2^{|Phi|}$
\begin{dem}{}
 $i : W^f \rightarrow 2^{SF(Phi)}$
 $|w| \mapsto Psi \in SF(Phi) / M, w\ models Psi_i$
est bien définie et injective.
\end{dem}

\begin{thm}{}
 $\forall \Psi \in SF(Phi)$,
$M,w \models \Psi$ ssi $M^f , |w| \models Psi$
\end{thm}
Dem : par induction

Logiques candidates pour la filtration : 
les logiques dont le problème SAT est NEXPTIME-complet
Pour celles là, la filtration donne un algo optimal.
 
\end{prop}
\end{document}
